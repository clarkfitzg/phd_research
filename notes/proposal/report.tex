% A simple template for LaTeX documents
% 
% To produce pdf run:
%   $ pdflatex paper.tex 
%


\documentclass[12pt]{article}

% Begin paragraphs with new line
\usepackage{parskip}  

% Change margin size
\usepackage[margin=1in]{geometry}   

% Graphics Example:  (PDF's make for good plots)
\usepackage{graphicx}               
% \centerline{\includegraphics{figure.pdf}}

% Allows hyperlinks
\usepackage{hyperref}

% Blocks of code
\usepackage{listings}
\lstset{basicstyle=\ttfamily, title=\lstname}
% Insert code like this. replace `plot.R` with file name.
% \lstinputlisting{plot.R}

% Monospaced fonts
%\usepackage{inconsolata}
% GNU \texttt{make} is a nice tool.

% Supports proof environment
\usepackage{amsthm}

% Allows writing \implies and align*
\usepackage{amsmath}

% Allows mathbb{R}
\usepackage{amsfonts}

% Numbers in scientific notation
% \usepackage{siunitx}

% Use tables generated by pandas
\usepackage{booktabs}

% Allows umlaut and non ascii characters
\usepackage[utf8]{inputenc}

% norm and infinity norm
\newcommand{\norm}[1]{\left\lVert#1\right\rVert}
\newcommand{\inorm}[1]{\left\lVert#1\right\rVert_\infty}

% Statistics essentials
\newcommand{\iid}{\text{ iid }}
\newcommand{\Exp}{\operatorname{E}}
\newcommand{\Var}{\operatorname{Var}}
\newcommand{\Cov}{\operatorname{Cov}}


%%%%%%%%%%%%%%%%%%%%%%%%%%%%%%%%%%%%%%%%%%%%%%%%%%%%%%%%%%%%

\begin{document}

\title{Project Proposal}
\date{\today}
\author{Clark Fitzgerald}
\maketitle

\begin{abstract}

\end{abstract}

\section{Statistical Calculations on GPU}
%%%%%%%%%%%%%%%%%%%%%%%%%%%%%%%%%%%%%%%%%%%%%%%%%%%%%%%%%%%%

\textbf{Develop strategies and tools to use GPU's for data analysis}

\subsection{Application 1 - Approximate Likelihood}

Ethan Anderes has been working on an application with Joseph Guinness to
use Vecchia's approximation to estimate the likelihood function of $X \sim
N(0, \Sigma)$ where $X$ is of high dimension, for example 1 million. The
approximation is used because the actual pdf involves $\Sigma^{-1}$,
requiring a Cholesky decomposition which is too expensive to compute
\cite{guinness2016permutation}.  The approximation can be posed as an
embarrassingly parallel problem.

The actual computations involve choosing strategies to chunk the data. 
My research goal is to understand and describe more generally the
GPU strategies that can be proven effective.

\subsection{Challenges}

This application will require understanding the computational model and
relevant programming tools for the GPU.

\subsection{Applications}

If a GPU can be used from a high level language to accelerate array
oriented computations by one or two orders of magnitude this would be very
interesting.

\section{Code Generation}
%%%%%%%%%%%%%%%%%%%%%%%%%%%%%%%%%%%%%%%%%%%%%%%%%%%%%%%%%%%%

\textbf{Automatically generate R bindings for existing C and C++ libraries}

This would be an extension of Temple Lang's
work with the RCIndex \cite{R-RCIndex} and RCodeGen \cite{R-RCodegen}
packages, along with other prior work.

Writing R / C bindings by hand has several problems: 
\begin{enumerate}
    \item{Error Prone:} Computers do repetitive tasks better than humans.
    \item{Tedious:} Nobody wants to spend their time doing this.
    \item{Versioning:} A minor change in a software dependency version ie. 2.1
        to 2.2 often breaks the bindings.
\end{enumerate}

This allows one to very quickly add new
capabilities to R as a system. The new capabilities then facilitate rapid
prototyping and novel types of data analysis, using R as a `glue language'.
Temple Lang describes further applications and motivations in the paper in
the RCIndex package repository \cite{R-RCIndex}.

\subsection{Related Work}

Many packages on CRAN simply wrap existing C/C++ libraries. When done
thoughtfully this can be very useful. Ie. the programmer maps the C/C++ API
into idiomatic data structures and calls them in the new language, in this
case R.  The intent with the code generation is to \emph{augment} rather
than \emph{replace} such libraries. 

The approach in RCIndex differs from existing software like SWIG
\cite{swig} and Rcpp \cite{R-Rcpp} because it hooks directly into the clang
compiler rather than processing the text itself. This means we have a code
analysis and generation tool that will be more robust and consistent with
behavior of a compiler. 

\subsection{Challenges}

This will require me to spend significant time building expertise in C/C++.

Memory management and garbage collection is potentially difficult. A C
routine might allocate memory that R knows nothing about. Then how is that
memory protected and freed? A possible solution to this problem is to
recursively look
through the body of the code itself, which RCIndex doesn't yet do.

\subsection{Applications}

The mature C++ computer vision library openCV
\cite{opencv_library} is an example of capabilities that we would like to
access from within R. It's possible to directly write computer vision
algorithms in R, but this is a huge amount of duplicated work if a mature
library already exists.

Another application is quickly providing R bindings to cutting edge
specialized machine learning code like Professor Cho-Jui Hsieh's Hogwild++
\cite{zhang2016hogwild}.

\newpage
\bibliographystyle{plain}
\bibliography{citations} 

\end{document}
