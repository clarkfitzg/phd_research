% A simple template for Beamer presentations in LaTeX
% 
% To produce pdf run:
%   $ pdflatex beamer.tex 
%

\documentclass{beamer}
\usetheme{Singapore}

% Bibliography
%\usepackage{cite}
\usepackage{biblatex}


\hypersetup{colorlinks=true}

% Show table of contents between sections
\AtBeginSection[]
{
  \begin{frame}
    \frametitle{Table of Contents}
    \tableofcontents[currentsection]
  \end{frame}
}

% Graphics examples
%\centerline{\includegraphics[height=2.5in]{figs/normal.pdf}}
%\includegraphics[width=4in]{figs/makefile.png}

%%%%%%%%%%%%%%%%%%%%%%%%%%%%%%%%%%%%%%%%%%%%%%%%%%%%%%%%%%%%

\begin{document}

\title{Parallel Computing Through Code Analysis}
\date{\today}
\date{10 May 2017}
\author{Clark Fitzgerald}
\institute{UC Davis - Statistics Student Seminar}

\frame{\titlepage}

\begin{frame}
    \frametitle{Outline}
    \tableofcontents
\end{frame}

%%%%%%%%%%%%%%%%%%%%%%%%%%%%%%%%%%%%%%%%%%%%%%%%%%%%%%%%%%%%
\section{Motivation}
%%%%%%%%%%%%%%%%%%%%%%%%%%%%%%%%%%%%%%%%%%%%%%%%%%%%%%%%%%%%
\begin{frame}

    \frametitle{Loop detectors count cars, measuring velocity and
    occupancy}

\centerline{\includegraphics[height=2.5in]{loop_detector.jpg}}

\end{frame}
%%%%%%%%%%%%%%%%%%%%%%%%%%%%%%%%%%%%%%%%%%%%%%%%%%%%%%%%%%%%
\begin{frame}

\frametitle{Caltrans Performance Measurement System (PeMS) records this
data for the whole state}

    \begin{itemize}
        \item Each sensor measures 3 quantities
        \item Every thirty seconds they produce a new data point
        \item $43,680$ sensors in California
    \end{itemize}

    $\implies$  377 million new data points per day

\end{frame}
%%%%%%%%%%%%%%%%%%%%%%%%%%%%%%%%%%%%%%%%%%%%%%%%%%%%%%%%%%%%
\begin{frame}

    \frametitle{Traffic engineers study the relationship between 
    density and flow \cite{li2011fundamental}}

\centerline{\includegraphics[height=2.5in]{fundamental_diagram.png}}

\end{frame}
%%%%%%%%%%%%%%%%%%%%%%%%%%%%%%%%%%%%%%%%%%%%%%%%%%%%%%%%%%%%
\begin{frame}

    \frametitle{Using more data allows new types of analyses}

    \begin{itemize}
        \item Finding groups of similar detectors
        \item Effect of policy such as speed limits and carpool lanes
        \item Impact of road features such as on/off ramps
        \item Effect of weather
    \end{itemize}

\end{frame}
%%%%%%%%%%%%%%%%%%%%%%%%%%%%%%%%%%%%%%%%%%%%%%%%%%%%%%%%%%%%
\begin{frame}

    \frametitle{Lets scale up the computations with parallel programming}

    \begin{itemize}
        \item Several parallel strategies exist to improve performance
        \item Which is best depends on the platform and data
        \item Each one requires deeper knowledge of that particular technology
        \item Therefore techniques to programmatically identify and use parallel
  patterns implicit in code would be valuable
    \end{itemize}

\end{frame}
%%%%%%%%%%%%%%%%%%%%%%%%%%%%%%%%%%%%%%%%%%%%%%%%%%%%%%%%%%%%
\section{Simple example}
%%%%%%%%%%%%%%%%%%%%%%%%%%%%%%%%%%%%%%%%%%%%%%%%%%%%%%%%%%%%
\begin{frame}


\frametitle{Topics}

\begin{itemize}

    \item Why I'm a bad programmer
    \item Introduction to scipy.stats (from Python!)
    \item Live Demo

\end{itemize}

\end{frame}
%%%%%%%%%%%%%%%%%%%%%%%%%%%%%%%%%%%%%%%%%%%%%%%%%%%%%%%%%%%%
\section{Conclusion}
%%%%%%%%%%%%%%%%%%%%%%%%%%%%%%%%%%%%%%%%%%%%%%%%%%%%%%%%%%%%
\begin{frame}


\frametitle{Topics}

\end{frame}
%%%%%%%%%%%%%%%%%%%%%%%%%%%%%%%%%%%%%%%%%%%%%%%%%%%%%%%%%%%%

\begin{frame}[t,allowframebreaks]
\frametitle{References}
\printbibliography
\end{frame}

\end{document}
