% A simple template for LaTeX documents
% 
% To produce pdf run:
%   $ pdflatex paper.tex 
%


\documentclass[12pt]{article}

% Begin paragraphs with new line
\usepackage{parskip}  

% Change margin size
\usepackage[margin=1in]{geometry}   

% Graphics Example:  (PDF's make for good plots)
\usepackage{graphicx}               
% \centerline{\includegraphics{figure.pdf}}

% subfigures, side by side
\usepackage{subcaption}

% hyperlinks
\usepackage{hyperref}

% Blocks of code
\usepackage{listings}
\lstset{basicstyle=\ttfamily, title=\lstname}
% Insert code like this. replace `plot.R` with file name.
% \lstinputlisting{plot.R}

% Monospaced fonts
%\usepackage{inconsolata}
% GNU \texttt{make} is a nice tool.

% Supports proof environment
\usepackage{amsthm}

% Allows writing \implies and align*
\usepackage{amsmath}

% Allows mathbb{R}
\usepackage{amsfonts}

% Numbers in scientific notation
% \usepackage{siunitx}

% Use tables generated by pandas
\usepackage{booktabs}

% Allows umlaut and non ascii characters
\usepackage[utf8]{inputenc}

% Insert blank pages
\usepackage{afterpage}
%\afterpage{\null\newpage}

% norm and infinity norm
\newcommand{\norm}[1]{\left\lVert#1\right\rVert}
\newcommand{\inorm}[1]{\left\lVert#1\right\rVert_\infty}

% Statistics essentials
\newcommand{\iid}{\text{ iid }}
\newcommand{\Exp}{\operatorname{E}}
\newcommand{\Var}{\operatorname{Var}}
\newcommand{\Cov}{\operatorname{Cov}}


%%%%%%%%%%%%%%%%%%%%%%%%%%%%%%%%%%%%%%%%%%%%%%%%%%%%%%%%%%%%

\begin{document}

\begin{center}
    \large Automatic Parallelism Through Code Analysis 

    \normalsize Clark Fitzgerald
\end{center}

\vspace{3\baselineskip}

For the 2017 Summer Research Application I plan to develop software capable
of transforming serial R programs into parallel. This software will take
serial code and a description of the data to be analyzed in as an input,
and output parallel code through multiprocessing.

Parallel opportunities can be found through analysis of base R's apply family of functions,
which include \texttt{lapply, apply, sapply, tapply, by, mapply, Map, vapply, outer
by, replicate}. These are all variants of the map reduce computational model which
has been successful for implementing large scale parallel systems
\cite{dean2008mapreduce}.

I will apply the code analysis to several test cases including:
\begin{enumerate}
    \item Statistical simulation
    \item Analysis of data by group
\end{enumerate}

Quantifying the gains and overhead associated with various forms of
parallelism is important

The following code:

\begin{verbatim}

apply(x, 1, 

\end{verbatim}

\bibliographystyle{plain}
\bibliography{../citations,../Rpackages} 

\end{document}
