% A simple template for LaTeX documents
% 
% To produce pdf run:
%   $ pdflatex paper.tex 
%

\documentclass[12pt]{article}

% Change margin size
\usepackage[margin=1in]{geometry}   


%%%%%%%%%%%%%%%%%%%%%%%%%%%%%%%%%%%%%%%%%%%%%%%%%%%%%%%%%%%%

\begin{document}

\section{Literature Review}


\subsection{SQL query optimization}

SQL is nice because the unit of analysis is a single query. Thus the query
optimizer only has to figure out how to do this one query efficiently,
although nested subqueries could add complexity.

\cite{Chaudhuri:1998:OQO:275487.275492} gives an overview of query
optimization for relational databases. The execution plan is a description
of the physical operators, ie. table scan, that takes one or more input
data streams and outputs another stream.

The author distinguishes Query optimization from execution. The former
decides the execution plan, the latter executes it.

Query optimization needs: a \textbf{search space} of possible solutions,
\textbf{cost estimation} for each solution, and an \textbf{enumeration
algorithm} to explore the different solutions.

System R was an early database technology that introduced some of the
query optimization techniques.

\subsubsection{relation to my work}

An R expression, particularly something like tables piped through dplyr, is
like a query. If we have an expensive vectorized function \texttt{f} and a
large data structure \texttt{x} then \texttt{f(head(x))} is semantically
equivalent to, but much cheaper than \texttt{head(f(x))}. The reason is
because the filtering happens first.

\end{document}
